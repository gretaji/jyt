\renewcommand{\baselinestretch}{1.5}
\chapter{Abstract}
\renewcommand{\baselinestretch}{\mystretch}

%\setlength{\parindent}{2em}

In this thesis, the aim is to manufacture a new printable filament with ABS pellets and pumice powder to manufacture 3D printed metal-pipe rectangular waveguides \cite{d20153}. As consideration attention has been paid to space exploration and manufacturing, there is no surprise that In-situ resource utilization becomes an urgent issue. The pumice dust could be regarded as a kind of lunar and martian regolith simulants. It has been demonstrated that Fused Deposition Modelling (FDM) could be applied in a micro-gravity environment\cite{jakus2017robust}. Therefore, it is valuable to utilise lunar and martian dust simulants in printable filament manufacturing for FDM technique. The most popular thermoplastic called Acrylonitrile butadiene styrene (ABS) is mixed with pumice as the base material. The various filament samples are manufactured with different proportions of ABS pellets and pumice powder. The composite filament samples and corresponding 3D-printed waveguides are investigated. This study conducts filament manufacturing by ABS and pumice (1-8 wt.\%). In fact, viable candidate materials for Ultimaker 2 printer are the filament samples made by 1-7 wt.\% pumice with ABS. There is a highly potential that the launch mass for space mission could be drastically decreased by using lunar and martian regolith for filament fabrication.\\
\\
\textit{Key words}: Lunar and Martian Dust; Acrylonitrile butadiene styrene (ABS); Fused Deposition Modelling (FDM); Waveguides








